\documentclass{article}
\usepackage[margin=1in]{geometry}
\usepackage{amsmath, amssymb}
\usepackage{color}
\newcommand{\bbR}{\mathbb{R}}
\newcommand{\bbC}{\mathbb{C}}
\newcommand{\rmP}{\mathrm{P}}
\newcommand{\rmE}{\mathrm{E}}
\newcommand{\calR}{\mathcal{R}}
\begin{document}
\begin{enumerate}
  \item
    Large deviations
    \begin{enumerate}
      \item
        Intuition \underline{($10$ minutes)}
        \begin{enumerate}
          \item\label{characteristic}
            Mention properties of characteristic/moment-generating function.
          \item
            Show $\log$-$\log$ plot comparing $p_n(\delta) = n^{-1}\sum_{k=1}^n X_k > \delta$ against appropriate CLT approximation; conclude radically different slopes for $\delta > \rmE X_1$.
          \item
            In light of preceding issue, motivate estimator $\hat p_n(\delta) = e^{-nI(\delta)}$ with a logarithmic first-order asymptotic:
            \[
              p_n(\delta) = e^{-nI(\delta) + o(n)},
            \]
            to conclude that finding exponential decay $I(\delta)$ involves finding the following limit.
            \[
              \lim_{n\rightarrow\infty} \frac1n\log p_n(\delta) = \lim_{n\rightarrow\infty} \Big( -I(\delta) + \frac{o(n)}{n} \Big) = -I(\delta)
            \]
          \item
            With preceding intuition, we can define a large deviations principle, noting that we extend from function $I: \bbR \rightarrow [0, \infty)$ to $r: \calR \rightarrow [0,\infty)$ by a variational expression $rA = \inf_{\delta \in A} I(\delta)$.
        \end{enumerate}
      \item
        Useful large deviations results \underline{($10$ minutes)}
        \begin{enumerate}
          \item
            State G\"artner-Ellis theorem; hint at gist of proof by mentioning Chebyshev and change of measure.
            Show how $\log$-$\log$ plot of LDP for preceding example is consistent with $p_n(\delta) = e^{-nI(\delta) + o(n)}$.
          \item\label{processes}
            State how stochastic processes can be viewed as measurable maps on the weak algebra generated by the time-vector projections, and that on this algebra, measures and convergence (with tightness) are determined by finite-dimensional projections.
            Ask loaded question of if similar trick can be done with large deviations results.
          \item
            State answer: Dawson-G\"artner theorem.
          \item
            State Friedlin-Wentzel theorem.
          \item
            State limitations of Friedlin-Wentzel (``bad'' diffusion coefficients).
        \end{enumerate}
      \item
        Large deviations applications (perhaps instead at end of talk, or beginning if we start instead with Affine processes) \underline{($8$ minutes)}
        \begin{quote}
          \color{red}
          Though we haven't researched in this direction, I would like to have such results, because:
          \begin{enumerate}
            \item
              I enjoy computer science, so I would like to implement the simulation/statistical algorithms.
            \item
              The statistical applications would be something I could leverage into a career.
            \item
              The particular statistical application of affine processes having a nice inverse problem `econometric data $\rightarrow$ affine process' seems very interesting and could be something in which the other committee members have expertise.
            \item
              The paper involving LDP's in stochastic volatility problems seems similar to JP's results from my naive perspective, and could be worth investigating for similar reasons.
          \end{enumerate}
        \end{quote}
    \end{enumerate}
  \item
    Affine processes \underline{($12$ minutes)}.
    \begin{enumerate}
      \item
        Define Markov processes and time-homogeneity.
        Show how these, together with \ref{characteristic} and \ref{processes}, are such that law is determined by the marginals' exponential moments, i.e.\ the function $M$:
        \[
          M(t, x, u) = \rmE_x\exp\langle u, X_t \rangle.
        \]
      \item
        State definition of an affine process.
      \item
        Show example of simple process which ends up being affine.
        \begin{align*}
          & X_t = x + \sigma W_t + \sum_{k=1}^\infty Z_k 1_{[0, t]}(T_k) \\
          &\leadsto \quad \rmE_x\exp\langle u, X_t \rangle = \exp\bigg( t\frac12\langle u, \sigma^T \sigma u \rangle + t\int_{\bbR \setminus\{0\}} \big(\exp\langle v, u \rangle - 1\big) \mu({\rm d}v) + \langle u, x \rangle\bigg)
        \end{align*}
        Make observation that $(\phi, \psi)$ satisfy a generalized Riccati ODE.
      \item
        State Keller-Ressel, Mayerhofer result for affine transform formula.
    \end{enumerate}
  \item
    Current results \underline{($15$ minutes)}
    \begin{enumerate}
      \item
        State with Kang, Kang result; recall limitations of Friedlin-Wentzel.
      \item
        Mention how this result does not include our earlier example, due to the discontinuities.
      \item
        Show how LDP is derived setting up $(X^\epsilon)_{\epsilon > 0}$ such that following reparameterizations hold,
        \[
          \epsilon\phi^\epsilon(t, u/\epsilon) = \phi(t, u), \quad \epsilon\psi^\epsilon(t, u/\epsilon) = \psi(t, u),
        \]
        and using G\"artner-Ellis with Dawson-G\"artner.
      \item
        Remark how we can do the preceding setup by setting up Riccati ODE's accordingly and using affine transform formula.
      \item
        Verify it all checks out on birth process.
      \item
        Verify it all checks out on Hawkes process.
    \end{enumerate}
  \item
    Potential further results \underline{($5$ minutes)}
    \begin{enumerate}
      \item
        Apply the result for a specific model.
      \item
        See if we can provide alternative proof with exponential martingales.
      \item
        Experiment with reparameterization that works with compensated measures / truncation functions, since method seems to work independently of these details.
    \end{enumerate}
\end{enumerate}
\end{document}
